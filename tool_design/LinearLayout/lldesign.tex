\documentclass[12pt]{article}
\usepackage[letterpaper,margin=1in]{geometry}
\usepackage[
  pdftitle={Visualizing the Longest Path Through an Assembly Graph},
  pdfauthor={Marcus Fedarko}
]{hyperref}

\begin{document}

\title{Visualizing the Longest Path Through an Assembly Graph}
\maketitle

\section{Background}

DNA assembly programs generate \emph{assembly graphs} as their output:
graphs in which each node represents a \emph{contig}, an unambiguously
contiguous region of DNA determined from sequencing data, and each edge
represents a possible overlap between two contigs. The challenge of the
processes undertaken after assembly, then---referred to as
\emph{scaffolding} and, finally, \emph{finishing}---are interpreting
this data to determine the correct path through the contigs that represents
the correct DNA sequence (\emph{genome}) of the organism being sequenced.

Modern visualization software of assembly graphs, including Bandage (by Wick
et al. 2015) and ABySS-Explorer (by Nielsen et al. 2009) takes the general
approach of ``showing everything at once'': presenting all the information
in an assembly graph to the user from the start. While broadly informative,
this approach can be confusing and difficult to interpret for many users,
particularly for
genomes containing a massive amount of contigs (e.g. for those of eukaryotic
organisms, or of metagenome assemblies).

The approach we plan to take is a bit different. Our focus is less on
displaying all information at once, but on displaying the most important
features: the \emph{longest paths} of contigs through the graph, separated
by connected components, with certain common patterns in sequencing data
highlighted for the user to observe. These features, along with many others,
will be contained in a fully interactive web-based visualization tool that
we hope will provide a novel way of examining assembly graph data.

\section{Longest Paths}

The problem of finding the longest path through a directed graph is
NP-Hard, provable by reduction from the Hamiltonian Cycle Problem.

\section{Highlighting Patterns in Assembly Data}

We highlight what Miller, Koren, and Sutton (2010) describe as ``bubbles,''
``frayed ropes,'' and ``spurs.'' (TODO -- highlight cycles, also.)

\end{document}
